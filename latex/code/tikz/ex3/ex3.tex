\documentclass{article}

\usepackage[tightpage, active]{preview}

\setlength{\PreviewBorder}{1pt} % the space between the final figure and the border

\usepackage{amsmath}

\usepackage[table]{xcolor}

\usepackage{graphicx}
\usepackage{caption}
\usepackage{subcaption}
\usepackage{hyperref}

%===========================================
%   Tikz
%-------------------------------------------
\usepackage{tikz}
\usetikzlibrary{%
  arrows,%
  backgrounds,%
  calc,%
  chains,%
  decorations.pathmorphing, % for snaking lines
  decorations.pathreplacing,%
  fit,%
  matrix,%
  positioning,%
  scopes,%
  shapes.geometric,%
  shapes,%
  shapes.symbols,%
  spy,%
  trees%
}

\usepackage{pgfplots}
%\pgfplotsset{compat=1.13}

% refer to
% ftp://ftp.rrzn.uni-hannover.de/pub/mirror/tex-archive/graphics/pgf/contrib/tikz-3dplot/tikz-3dplot_documentation.pdf
\usepackage{tikz-3dplot}

%===========================================
%   preview environment
%   this should be added after all packages
%-------------------------------------------
\PreviewEnvironment{forest}
\PreviewEnvironment{tikzpicture}
\PreviewEnvironment{equation}
\PreviewEnvironment{equation*}
\PreviewEnvironment{tabular}

%===========================================
%  TikZ settings 
%-------------------------------------------
\tikzstyle{my arrow}=[%
  ->,
  >=stealth',
  shorten >=1pt % if enabled, the arrow head will not touch the edges
]


\begin{document}

%==============================
% Page 0
%
% curved path with two control points
%------------------------------
\begin{tikzpicture}
% there are several ways to specify the radius of a circle
\draw (0, 0) .. controls (1, 1) and (2, 1) .. (2, 0);
\draw[thin, densely dashed] (0, 0) -- (1, 1) (2, 1) -- (2, 0);
\filldraw[gray] (0, 0) circle [radius=2pt]
                (1, 1) circle (2pt)
                (2, 1) circle (2pt)
                (2, 0) circle [x radius=2pt, y radius=2pt];
\end{tikzpicture}

%==============================
% Page 1
%
% curved path with one control point
% In this case, the first one is used twice
%------------------------------
\begin{tikzpicture}
\draw (0, 0) .. controls (1, 1) .. (2, 0);
\draw[densely dashed] (0, 0) -- (1, 1) -- (2, 0);
\filldraw[gray] (1, 1) circle (2pt);
\end{tikzpicture}

%==============================
% Page 2
%
% Use control points to draw a circle
%------------------------------
\begin{tikzpicture}
\draw (-1.5, 0) -- (1.5, 0);
\draw (0, -1.5) -- (0, 1.5);
\draw (-1, 0) .. controls (-1, 0.5) and (-0.5, 1) .. (0, 1);
\draw (1, 0) arc [start angle=0, end angle=90, radius=1];
\end{tikzpicture}

%==============================
% Page 3
%
% Draw a grid
%------------------------------
\begin{tikzpicture}
\draw[gray, step=0.5cm, very thin] (-1.4, -1.4) grid (1.4, 1.4);
\draw (-1.5, 0) -- (1.5, 0);
\draw (0, -1.5) -- (0, 1.5);
\draw (0, 0) circle [radius=1];
\end{tikzpicture}

%==============================
% Page 4
%
% Draw a grid with styles
% Note the styles are defined inside the environment
%------------------------------
\begin{tikzpicture}[
  my help lines/.style={step=0.5cm, gray, very thin},
  my color/.style={color=blue!50},
]
\draw[my help lines, my color] (-1.4, -1.4) grid (1.4, 1.4);
\draw (-1.5, 0) -- (1.5, 0);
\draw (0, -1.5) -- (0, 1.5);
\draw (0, 0) circle (1cm);
\end{tikzpicture}

%==============================
% Page 5
%
% Draw a grid with global styles
%------------------------------
% Note that \tikzset is usally used outside of \begin{document}
% \tikzstyle is deprecated, use \tikzset instead!
%
% \tikzstyle{my help lines} = [step=0.5cm, gray, very thin];
% \tikzstyle{my color} = [color=red!50];
\tikzset{my help lines/.style={step=0.5cm, gray, very thin}};
\tikzset{my color/.style={color=blue!50}}
\begin{tikzpicture}
\draw[my help lines, my color] (-1.4, -1.4) grid (1.4, 1.4);
\draw (-1.5, 0) -- (1.5, 0);
\draw (0, -1.5) -- (0, 1.5);
\draw (0, 0) circle (1cm);
\end{tikzpicture}

%==============================
% Page 6
%
% Clipping
%------------------------------
\begin{tikzpicture}
% \clip[draw]
% \draw[clip]
% \path[clip, draw]
\clip (-0.1, -0.2) rectangle (1.1, 0.75);
\draw (-1.5, 0) -- (1.5, 0);
\draw (0, -1.5) -- (0, 1.5);
\draw (0, 0) circle (1cm);
\end{tikzpicture}

%==============================
% Page 7
%
% curves
%------------------------------
\begin{tikzpicture}
\draw[help lines] (0, 0) grid (2, 2);
\draw (0, 0) parabola (2, 2);
\end{tikzpicture}

%==============================
% Page 8
%
% curves
%------------------------------
\begin{tikzpicture}
\draw[help lines] (0, 0) grid (2, 2);
\draw (0, 0) parabola bend (1, 1) (2, 0);
\draw (1, 1) circle (1pt);
\end{tikzpicture}

%==============================
% Page 9
%
% Specifying coordinates
%
% +(0.5, 0) will not update the cursor!
%------------------------------
\begin{tikzpicture}
\draw (0, 0) -- +(0.5, 0) -- +(0.5, 0.5) -- cycle;
\end{tikzpicture}

%==============================
% Page 10
%
% Specifying points
% ++(0.5, 0) updates the cursor!
%------------------------------
\begin{tikzpicture}
\draw (0, 0) -- ++(0.5, 0) -- ++(0, 0.5) -- cycle;
\end{tikzpicture}

%==============================
% Page 11
%
% Specifying points
% Using polar coordinates
% (degree:radius)
%------------------------------
\begin{tikzpicture}
\draw (0, 0) -- (0:0.5) -- +(90:0.5) -- cycle;
\end{tikzpicture}

%==============================
% Page 12
%
% Specifying points by crossing
%------------------------------
\begin{tikzpicture}
\draw (0, 0) -- (45:0.5) -- (45:0.5 |- 0,0);
\end{tikzpicture}

%==============================
% Page 13
%
% Specifying points by crossing
% Here it uses |-
% a vertical line passing (0.5, 0.5)
% a horizontal line passing (0, 0)
%------------------------------
\begin{tikzpicture}
\draw (0, 0) -- (0.5, 0.5) -- (0.5,0.5 |- 0,0);
\end{tikzpicture}

%==============================
% Page 14
%
% Specifying points by crossing
% Here it uses -|
% a horizontal line passing (0, 0.5)
% a vertical line passing (0.5, 0)
%------------------------------
\begin{tikzpicture}
\draw (0, 0) -- (0, 0.5) -- (0,0.5 -| 0.5,0);
\end{tikzpicture}

%==============================
% Page 15
%
% Specifying points by intersection
% Note the two line **segments** must have an intersection
% Otherwise, it will cause an error when compiling
%------------------------------
\begin{tikzpicture}
% The path command is not visible in the figure, but it still occupies space!!!
\path[name path=line 1] (0, 0.5) -- (0.5, 0.5);
\path[name path=line 2] (0.5, 0) -- (0.5, 0.51);

% x is the intersection of the two line segments: line 1 and line 2
\draw[name intersections={of=line 1 and line 2, by=x}] (0, 0) -- (0, 0.5) -- (x);
\end{tikzpicture}

%==============================
% Page 16
%
% scope
%------------------------------
%
% the options for the environment tikzpicture
% is applicable to everything inside
\begin{tikzpicture}[ultra thick]
\draw (0, 0) -- (0, 0.5);  % ultra thick by default
\begin{scope}[thin]
  \draw (0.5, 0) -- (0.5, 0.5); % it is thin
  \begin{scope}[dashed]
    \draw (1.0, 0) -- (1.0, 0.5); % thin and dashed
  \end{scope}
\end{scope}

\draw (0, 0.5) -- (1, 0.5); % ultra thick
\end{tikzpicture}

%==============================
% Page 17
%
% \foreach
%------------------------------
\begin{tikzpicture}
\foreach \x in {0.1, 0.2, 0.3} {
  \draw (0, 0) -- (1, \x);
}
\end{tikzpicture}

%==============================
% Page 18
%
% \foreach
%------------------------------
\begin{tikzpicture}
\draw (0, 0) -- (1, 0);
\foreach \x in {0, 0.5, 1.0} {
  \draw (\x, -1pt) -- (\x, 1pt);
}
\end{tikzpicture}

%==============================
% Page 19
%
% \foreach
%------------------------------
\begin{tikzpicture}
\draw (0, 0) -- (1, 0);
\foreach \x in {0, 0.5, 1.0} {
  % note we have to use cm here!
  \draw[xshift=\x cm] (0, -1pt) -- (0, 1pt);
}
\end{tikzpicture}

%==============================
% Page 20
%
% \foreach
%------------------------------
\begin{tikzpicture}
% the default step is 1
\foreach \x in {1, ..., 3} {
  % note we have to use cm here!
  \draw (\x, 0) circle (0.25 cm);
}
\end{tikzpicture}

%==============================
% Page 21
%
% \foreach
%------------------------------
\begin{tikzpicture}
% the step is 1.5 - 1 = 0.5
% the step is the second number minus the first number
\foreach \x in {1, 1.5, ..., 3} {
  % note we have to use cm here!
  \draw (\x, 0) circle (0.25 cm);
}
\end{tikzpicture}

%==============================
% Page 22
%
% \foreach
%------------------------------
\begin{tikzpicture}
% the step is 1.5 - 1 = 0.5
% the step is the second number minus the first number
\foreach \x in {1, 1.5, ..., 3, 4, 4.5, ..., 6} {
  % note we have to use cm here!
  \draw (\x, 0) circle (0.25 cm);
}
\end{tikzpicture}

%==============================
% Page 23
%
% \foreach
% two variables
%------------------------------
\begin{tikzpicture}
\foreach \x/\y in {1/0.5, 2/9, 3/10} {
  \draw (\x, 0) +(0.5, 0.5) rectangle +(-0.5, -0.5);
  % Note how a node is drawn!
  \draw (\x, 0) node{\y};
}
\end{tikzpicture}

%==============================
% Page 24
%
% node
% See https://tex.stackexchange.com/questions/230224/how-to-change-the-background-color-in-tikz
% for adding background
%
% left,right is more powerful than anchor=west, anchor=east
%------------------------------
\begin{tikzpicture}[background rectangle/.style={fill=olive!45}, show background rectangle]
\draw (0, 0) circle (2pt);
\draw (0, 0) node [anchor=north] {north};

\draw (1, 0) circle (2pt);
\draw (1, 0) node [right] {right};

\draw (3, 0) circle (2pt);
\draw (3, 0) node [left=4pt, fill=white] {left};

\draw (4, 0) circle (2pt) node [anchor=west] {west};

\draw (0, 1) -- node{n} (1, 1);
\draw (2, 1) -- node[above]{n} (3, 1);
\draw (4, 1) -- node[near start, below]{n} (5, 1);
\draw (0, -1) -- node[near end, above]{n} (1, -1);

\draw (1, 1) -- node{n} (2, 2);
\draw (2, 1) -- node[sloped, above]{n} (3, 2);
\draw (4, 1) -- node[sloped, below, near end]{n} (5, 2);
\end{tikzpicture}

\end{document}

